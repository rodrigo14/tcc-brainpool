\chapter{Fundamentação Teórica}\label{cap:fundamentacao}
Neste capítulo serão mostrados os conceitos matemáticos necessários para a geração desses parâmetros. 

%
% ARITMÉTICA MODULAR
%
\section{Aritmética Modular} \label{sec:aritmeticamodular} % RAMOS
Na Teoria dos Números \cite{Niven:2014} (ramo da matemática pura que estuda os números inteiros), uma importante ferramenta é a aritmética modular,  a qual estuda as propriedades de conjuntos formados a partir dos restos da divisão de números inteiros por um inteiro $n$, denominado módulo.
\par Uma congruência é  uma relação entre dois números inteiros que, quando divididos por um mesmo número (chamado de módulo da congruência), deixam o mesmo resto.  Em termos formais, dois números inteiros \(a\) e \(b\) são congruentes módulo \(n\) se n divide a diferença \(a - b\).  Esta relação pode ser expressa pela Equação \ref{eq:1}.

\begin{equation}
  a \equiv b \pmod n \label{eq:1}
\end{equation}

Para um inteiro positivo \(n\) e \(a\), \(b\), \(c\) inteiros quaisquer, a congruência tem as propriedades listadas nas Equações \ref{eq:2}, \ref{eq:3} e \ref{eq:4}.
\begin{eqnarray}
  a + c \equiv + c \pmod n \label{eq:2} \\
  a - c \equiv - c \pmod n \label{eq:3} \\
  a \times c \equiv \times c \pmod n \label{eq:4}
\end{eqnarray}

Em relação a divisão, ela não está definida em todos os casos. Se \(d\) é um inteiro que divide \(a\) e \(b\), então vale a relação apresentada na Equação \ref{eq:5}, onde \((n, d)\) é o maior divisor comum entre \(n\) e \(d\).

\begin{equation}
  \frac{a}{d} \equiv \frac{b}{d}\left(\mbox{mod}\ \frac{n}{(n,d)}\right) \label{eq:5}%\pmod {\frac{n}{(n, d)}}
\end{equation}

O inverso multiplicativo de a modulo \(n\), isto é, um inteiro \(b\) tal que \(a \times b = 1 \pmod  n\), existe apenas quando \((a, n) = 1\) (isto é, $a$ e $n$ são coprimos). O valor de \(b\) pode ser obtido através do Algoritmo de Euclides Estendido \cite{Halim:2013}.
\par A exponenciação modular calcula o resto de um número inteiro \(b\) quando elevado à um número inteiro \(k\) e é dividído por um inteiro positivo \(m\). Escolhidos a base \(b\), o expoente \(k\) e o módulo \(m\), uma exponenciação modular \(c\) é dada pela Equação \ref{eq:6}.

\begin{equation}
  c \equiv b^{k}\pmod m \label{eq:6}
\end{equation}

%
% ESTRUTURAS ALGÉBRICAS
%
\section{Estruturas Algébricas}

%
% OPERAÇÕES BINÁRIAS
%
\subsection{Operações Binárias}
Seja \(L\) um conjunto não vazio. Uma operação  sobre \(L\) é dita binária se ela possui dois elementos do conjunto \(L\) e os associa a um terceiro elemento em \(L\), sendo que opera em \(L\) para cada par \((a,b) \in L\). Uma operação binária sobre \(L\) é grafada conforme mostrado na Equação \ref{eq:7}.
\begin{equation}
  \varphi: L \times L \to L \label{eq:7}
\end{equation}

Uma operação binária é:
\begin{enumerate}
  \item \textit{Fechada}, se e somente se \( \forall a,b,c  \in L, (a \varphi b) \in L \);
  \item \textit{Comutativa}, se \( \forall a,b \in L , (a \varphi b) = (b \varphi a)\);
  \item \textit{Associativa}, se \( \forall a,b,c \in L, (a \varphi b) \varphi c = a \varphi (b \varphi c)\).
\end{enumerate}

Uma operação binária possui um elemento neutro \(e \in L\) se \( \forall a \in L, e \varphi a = a \varphi e = a\).

\par Dado um elemento \(a \in L\), \(a\) é invertível em relação à operação \( \varphi \) se existe  um elemento \(a' \in L\) que satisfaz \(a \varphi a' = a' \varphi a = e\). 

Um exemplo de operação binária é a soma no conjunto dos números naturais, como mostrado na Equação \ref{eq:8}.
\begin{equation}
  +: \mathbb{N} \times \mathbb{N} \to \mathbb{N} \label{eq:8}
\end{equation}

A imagem de um par \((a,b) \in \mathbb{N} \times \mathbb{N}\) pela operação \(+\) é denotada por \(a + b\) e é chamada de soma de \(a\) e \(b\).

%
% MONÓIDES E GRUPOS
%
\subsection{Monóides e Grupos}
Seja \(L\) um conjunto não vazio. Uma estrutura algébrica \((L, \varphi )\) (composta por um conjunto não vazio e uma operação binária) é um monóide se satisfaz as seguintes propriedades:

\begin{enumerate}
  \item a operação \( \varphi \) tem um elemento neutro em \(L\);
  \item a operação \( \varphi \) é associativa.
\end{enumerate}

Um grupo é um \textit{monóide} \((G,  \varphi)\) que atende também a propriedade:
\begin{enumerate}
\setcounter{enumi}{2}
\item todo elemento \(g \in G\) é invertível em relação a operação \(\varphi\), isto é, todo elemento \(g\) de \(G\) possui um elemento simétrico \(g'\) tal que \((g \varphi g') = (g' \varphi g) = e \), onde \(e\) é o elemento neutro da operação \(\varphi\).
\end{enumerate}
Um grupo é dito \textit{abeliano} se a operação for comutativa.

%
% ANÉIS E CORPOS
%
\subsection{Anéis e Corpos}
 Seja \(L\) um conjunto não vazio, e \(\sigma, \pi:L \times L \to L\) duas operações binárias em \(L\) denominadas, respectivamente, operações de adição e produto. A título de notação, será utilizada a expressão \(a+b\) como equivalente a \(\sigma(a,b)\) e \(a \times b\) como equivalente a \(\pi(a,b)\). O terno ordenado \((L, +,\times )\) é um anel se:
\begin{enumerate}
  \item \((L, +)\) é um grupo abeliano;
  \item \((L, +)\) possui um elemento neutro $0$, \textit{i.e}, \(\forall a \in A, a + 0 = 0 + a = a\);
  \item \((L, +)\) possui elementos simétricos para todos elementos de $L$, \textit{i.e}, \(\forall a \in A, \exists k \in A: a + k = 0\);
  \item \((L, \times)\) é um monóide;
  \item \((L, \times)\) é associativo, \textit{i.e}, \(\forall a,b,c \in L, (a \times b) \times c = a \times (b \times c)\);
  \item \((L, +, \times)\) são distributivos, \textit{i.e}, \(\forall a,b,c \in A, a \times (b + c) = a \times b + a \times c\) e \((b + c) \times a = b \times a + c \times a\);
\end{enumerate}

Um corpo é um anel cuja operação produto é comutativa e onde todos os elementos diferentes de zero (elemento neutro da adição) possuem um elemento inverso para a operação produto. Formalmente, um anel comutativo \(P\) é chamado de \textit{corpo} se atende à Equação \ref{eq:inv}.
\begin{equation}
  \forall a \in P \neq {0}, \exists a' \Rightarrow a \times a' = 1
    \label{eq:inv}
\end{equation}

A \textit{característica} de um corpo \(F\) é definido como o menor número de vezes em que é necessário usar o elemento identidade da multiplicação em uma soma para obter o elemento identidade da soma. Um corpo tem característica zero se a soma nunca alcançar a identidade da soma.

%
% VARIEDADES E CURVAS ALGÉBRICAS
% 
\subsection{Variedades e Curvas Algébricas}
Uma \textit{variedade algébrica} é o conjunto de soluções de um sistema de equações polinomiais. Seja \(K\) um corpo fechado, o anel de polinômios \(K[T_1 , T_2, ..., T_n ]\) nas \(n\) variáveis \(T_1 , T_2 , ..., T_n \) com coeficientes em \(K\) e \(\{f_{i}\}\) uma família de polinômios do anel. É denominada variedade algébrica \(V\) o subconjunto \(K_n\) formado pelos pontos que anulam (zeros) todos os polinômios da família \(\{f_{i}\}\), como apresentado na Equação \ref{eq:10}.
\begin{equation}
V = \{(x_1 , ..., x_n ) | f_{i}(1, ..., n) = 0; i = 1, ..., n\} \subseteq K \label{eq:10}
\end{equation}

A dimensão de uma variedade algébrica \(V\)  é o conjunto de funções racionais independentes que existem em \(V\).
\par Formalmente,  denomina-se \textit{curva algébrica} uma variedade algébrica de dimensão um. Esse conjunto de curvas descreve as figuras geométricas resultantes de uma seção cônica, a saber: círculos, hipérboles e eclipses \cite{Thomas:2004}.

%
% CURVAS ELÍPTICAS
%
\section{Curvas Elípticas}

%
% CRIPTOGRAFIA
%
\section{Criptografia} \label{sec:criptografia}
Criptografia é a ciência que trata de cifrar a escrita, de modo a torná-la ininteligível para os que não tenham os métodos convencionados para ter acesso a ela. Em Tecnologia da Informação, esta definição é ampliada a fim de englobar não só a escrita, mas qualquer tipo de documento ou dados que devam ser tratados secretamente. Um Sistema de Criptografia define um sistema em que um texto (ou documento, ou qualquer tipo de dado) é transformado através da criptografia em um texto cifrado (ciphertext) ou o texto cifrado é transformado de volta à informação original, através de algoritmos. A primeira ação é denominada cifragem ou criptografar, e a segunda é chamada de decifração ou decriptação. \cite{Portnoi:2005}

%
% CRIPTOGRAFIA SIMÉTRICA
%
\subsection{Criptografia simétrica}

%
% CRIPTOGRAFIA ASSIMÉTRICA
%
\subsection{Criptografia assimétrica}

%
% CRIPTOGRAFIA DE CURVAS ELÍPTICAS
%
\section{Criptografia de curvas elípticas}
