\begin{resumo}
O avanço da tecnologia nos últimos anos tem causado um grande impacto para a humanidade. Hoje em dia é praticamente inconcebível que exista alguma área que não utilize a tecnologia de alguma forma. Com esse crescimento tecnológico, torna-se imprescindível a criação de meios para garantir a segurança dos indivíduos. A criptografia tem importância central nesse contexto, pois evita que dados confidênciais sejam descobertos por pessoas mal-intencionadas. Porém, com o avanço da tecnologia, os sistemas criptográficos existentes demandam cada vez mais processamento e complexidade para se manterem seguros. Daí, é necessário buscar alternativas que garantam a segurança dos indivíduos no meio tecnológico e que utilizem menos recursos computacionais. Uma dessas alternativas é a utilização de curvas elípticas em criptografia. Essa alternativa oferece a mesma segurança que a criptografia baseada no RSA, porém, consumindo bem menos recursos computacionais. A segurança da criptografia de curvas elípticas é baseada na dificuldade em resolver o problema do logaribmo discreto. Porém, existem algoritmos computacionais que resolvem esse problema, comprometendo a segurança  do sistema criptográfico. O foco deste trabalho é explorar um desses algoritmos, a saber, o algoritmo de Pollard Rho.

 \vspace{\onelineskip}
    
 \noindent
 \textbf{Palavras-chaves}: Curvas elípticas. Criptografia. Pollard Rho. Logaritmo discreto.
\end{resumo}
