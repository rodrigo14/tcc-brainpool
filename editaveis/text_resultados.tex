\chapter{Resultados}
Neste capítulo serão apresentados os resultados obtidos pelo algoritmo Pollard-rho e suas variações juntamente com a descrição dos parâmetros utilizados.

\section{Descrição dos experimentos}
\begin{enumerate}

\item Implementar os seguintes algoritmos para solucionar o problema do logaritmo discreto para curvas elípticas:
  \begin{itemize}
  \item Pollard-rho com único processador.
  \item Pollard-rho com paralelização.
  \item Pollard-rho com múltiplos processadores.
  \end{itemize}
\item Escolher curvas com parâmetros mais simples cuja criptografia possa ser ``quebrada'' em um tempo adequado para que as comparações entre os diversos algoritmos possam ser realizadas.
\item Escolher pontos \(P\) e \(Q\) pertencentes à curva e elaborar o problema do logaritmo discreto.
\item Solucionar o problema utilizando os algoritmos implementados.
\item Verificar o tempo de execução necessário para cada algoritmo resolver o problema.
\item Comparar o tempo de execução dos algoritmos entre si.
% \item Verificar se os resultados obtidos condizem com os resultados esperados de acordo com a bibliografia pesquisada.

\end{enumerate}

\section{Implementação e execução dos algoritmos}
Para realizar a implementação do Pollard-rho com único processador (serial) seguiu-se a descrição do pseudo-algoritmo no Anexo \ref{alg:single} juntamente com a teoria explanada na seção \ref{sec:single}. Foi implementado uma versão modificada do algoritmo serial para que o código pudesse executar de forma paralela, afim de poder utilizar todos os processadores disponíveis da máquina. Essa medida obteve melhorias nos tempos de execução, os quais serão apresentados na seção \ref{sec:execution_time}.

O algoritmo mais eficiente, em termos de tempo de processamento, foi o Pollard-rho com múltiplos processadores (multiprocessado), como já esperado na seção \ref{sec:parallelized}. A utilização de um servidor central para o armazenamento de pontos é o ponto positivo do algoritmo, pois o processo que calcula os pontos ganha em velocidade em não ter a responsabilidade de checar a colisão de pontos. O ponto negativo dessa abordagem é que o servidor não pode armazenar infinitos pontos, pois existe o limite de memória RAM. Para isso utilizou-se a propriedade distintiva (seção \ref{sec:distinguished}).

\section{Tempos de execução}
\label{sec:execution_time}

\begin{table}[]
\centering
\label{table:results}
\begin{tabular}{|l|c|c|c|}
\hline
\rowcolor[gray]{0.9}
\multicolumn{1}{|c|}{\(bits\)} & \textbf{Serial} & \textbf{Paralelizado} & \textbf{Multiprocessado} \\ \hline
32                             & 0h00m01s        & 0h00m01s              & 0h00m01s                 \\ \hline
36                             & 0h00m08s        & 0h00m04s              & 0h00m01s                 \\ \hline
40                             & 0h00m46s        & 0h00m19s              & 0h00m03s                 \\ \hline
44                             & 0h02m58s        & 0h01m26s              & 0h00m14s                 \\ \hline
48                             & 0h17m11s        & 0h14m47s              & 0h01m04s                 \\ \hline
52                             & 0h20m26s        & 0h16m49s              & 0h04m54s                 \\ \hline
56                             & 1h27m28s        & 2h46m18s              & 0h29m12s                 \\ \hline
60                             & 15h11m05s       & 8h56m45s              & 0h41m38s                 \\ \hline
64                             & 2d 14h37m58s    & 1d 15h52m12s          & 6h44m23s                 \\ \hline
65                             & 3d 20h45m48s    & -                     & 10h33m34s                \\ \hline
66                             & 6d 04h30m10s    & -                     & -                        \\ \hline
\end{tabular}
\caption{Resultado dos algoritmos Pollard-rho}
\end{table}
