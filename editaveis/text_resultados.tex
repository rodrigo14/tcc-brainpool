\chapter{Resultados}
Para colocar a teoria à prova, foram feitos diversos testes

\section{Descrição dos experimentos}
\begin{enumerate}

\item Implementar os seguintes algoritmos para solucionar o problema do logaritmo discreto para curvas elípticas:
  \begin{itemize}
  \item Pollard-rho com único processador.
  \item Pollard-rho com múltiplos processadores.
  \end{itemize}
\item Escolher uma ou mais curvas com parâmetros mais simples (a definir as curvas que serão utilizadas) cuja criptografia possa ser ``quebrada'' em um tempo adequado para que as comparações entre os diversos algoritmos possam ser realizadas.
\item Escolher os pontos \(P\) e \(Q\) pertencentes à curva e elaborar o problema do logaritmo discreto.
\item Solucionar o problema utilizando os algoritmos implementados.
\item Verificar o tempo de execução necessário para cada algoritmo resolver o problema.
\item Comparar o tempo de execução dos algoritmos entre si.
\item Verificar se os resultados obtidos condizem com os resultados esperados de acordo com a bibliografia pesquisada.

\end{enumerate}
