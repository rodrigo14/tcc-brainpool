\chapter{Metodologia}
Metodologias específicas colaboram para estabelecer diretrizes e boas práticas na condução do trabalho, conferindo padronização, noções de pesquisa científica, dentre outras contribuições. \cite{Wohlin:2000}

Com o intuito de guiar a pesquisa de forma adequada, esse capítulo aborda sobre os diversos tipos de metodologias de pesquisa, de modo a definir qual se adequa melhor ao projeto.

%
% CLASSIFICAÇÃO DA PESQUISA
%
\section{Classificação da pesquisa}
A seguir serão apresentados os grupos de classificação de pesquisa, quanto à natureza da pesquisa, abordagem do problema, objetivos de uma pesquisa e procedimentos técnicos.

\begin{itemize}
\item Natureza da pesquisa:
    \begin{itemize}
    \item \textbf{Pesquisa básica:} Possui o objetivo de gerar novos conhecimentos para ciência. Neste tipo de pesquisa não é obrigatório que o conhecimento gere um uso prático \cite{Tafner:2007}.
    \item \textbf{Pesquisa Aplicada:} Visa gerar uma maior compreensão para assuntos práticos dirigidos à solução de problemas específicos \cite{Tafner:2007}.
    \end{itemize}
\item Abordagem do problema:
    \begin{itemize}
    \item \textbf{Pesquisa Quantitativa:} O estudo quantitativo considera que tudo pode ser quantificável, ou seja, o estudo que pode ser traduzido em números e requer uso de técnicas estatísticas para sua análise \cite{Travassos:2002}.
    \item \textbf{Pesquisa Qualitativa:} O estudo qualitativo é descritivo, sendo assim, não pode ser analisada de forma mensurável e sim indutivamente \cite{Travassos:2002}.
    \end{itemize}
\item Objetivos de uma pesquisa:
    \begin{itemize}
    \item \textbf{Pesquisa Exploratória:} é utilizada pelo pesquisador para se familiarizar com um assunto pouco explorado. Ao decorrer ou no final da pesquisa exploratória, o pesquisador poderá estar apto para formular hipóteses \cite{Gil:2008}.
    \item \textbf{Pesquisa Descritiva:} é usada quando se tem um conhecimento do assunto e se quer descrever um fenômeno. Hipóteses podem ser formuladas com base em conhecimentos prévios, procurando confirmá-las ou negá-las \cite{Gil:2008}.
    \item \textbf{Pesquisa Explicativa:} é classificada com base em procedimentos técnicos, podendo ser quantitativa ou qualitativa. A pesquisa quantitativa traduz em números os estudos realizados e se utiliza técnicas estatísticas para comprovar os fatos \cite{Gil:2008}.
    \end{itemize}
\item Procedimentos técnicos:
    \begin{itemize}
    \item \textbf{Pesquisa bibliográfica:} é realizada a partir do levantamento de referências teóricas sobre o tema. De forma geral, toda pesquisa inicia-se como uma pesquisa bibliográfica, mas há aquelas que dependem exclusivamente desse tipo de pesquisa. Em essência, a conclusão desse tipo de pesquisa é uma compilação das publicações referentes ao tema \cite{Tafner:2007}.
    \item \textbf{Pesquisa documental:} semelhante à pesquisa bibliográfica, diferencia-se pela natureza das fontes. Tem como base documentos sem tratamento analítico, como tabelas, cartas, fotos, pinturas, dentre outros \cite{Gil:2008}.
    \item \textbf{Pesquisa experimental:} seleciona grupos de assuntos coincidentes e submete-os a tratamentos diferentes, com isso, ocorre uma verificação se há variáveis estranhas e checa-se as diferenças observadas nas respostas e se são estatisticamente significantes \cite{Tafner:2007}.
    \item \textbf{Pesquisa \textit{ex-post facto}:} a tradução literal é ``de um fato passado'', ou seja, a pesquisa \textit{ex-post facto} é realizada após a ocorrência de variáveis no objeto de estudo. A principal característica deste tipo de pesquisa é o fato de os dados serem coletados após a ocorrência dos eventos. Assim, investiga possíveis relações de causa e efeito entre um determinado fato identificado pelo pesquisador e um fenômeno que ocorre posteriormente \cite{Gil:2008}.
    \item \textbf{Levantamento (\textit{survey}):} é uma investigação realizada em retrospecto, que em seguida, mediante análise quantitativa, chega às conclusões correspondentes aos dados coletados. O levantamento feito com informações de todos os integrantes do universo da pesquisa origina um censo \cite{Travassos:2002}.
    \item \textbf{Estudo de caso:} o estudo de caso busca o aprofundamento nas questões propostas, estudando um único grupo ou comunidade, utilizando mais a observação direta do que a interrogação, captando as explicações e interpretações do que ocorre no grupo. No estudo de campo, o pesquisador está inserido no grupo, para poder entender melhor as regras, os costumes e as convenções \cite{Travassos:2002}.
    \end{itemize}
\end{itemize}

%
% METODOLOGIA DE PESQUISA
%
\section{Metodologia de pesquisa}
Analisando-se o tema proposto, observa-se uma natureza de pesquisa aplicada. Este trabalho de conclusão de curso visa a implementação do algoritmo Pollard-rho e suas variações, carregando consigo diversos conhecimentos de assuntos práticos e gerando uma solução para um problema específico, no caso, o problema do logaritmo discreto no contexto de curvas elípticas.

Considerando os objetivos de estudo deste trabalho, foi incorporada uma abordagem explicativa afim de apresentar valores quantitativos que demonstrem a relação entre os algoritmos implementados e suas diferenças, além de uma análise dos dados e resultados utilizando de técnicas com o intuito de comprovar fatos inerentes à segurança de curvas elípticas.

Dessa forma, este trabalho classifica-se como uma pesquisa:
\begin{itemize}
\item \textbf{Quanto à natureza}, este trabalho classifica-se como uma \textbf{pesquisa aplicada}.
\item \textbf{Quanto aos objetivos da pesquisa}, este trabalho pode ser classificado como uma \textbf{pesquisa explicativa}.
\item \textbf{Quanto à abordagem do problema}, este trabalho pode ser classificado como uma \textbf{pesquisa quantitativa}.
\item \textbf{Quanto aos procedimentos técnicos}, foram utilizados a \textbf{pesquisa bibliográfica} e a \textbf{pesquisa experimental}.
\end{itemize}

%
% ATIVIDADE DE PESQUISA
%
\section{Atividades de projeto}
Os seguintes pacotes de trabalho, não necessariamente nessa ordem, são fundamentais para compor este projeto de monografia:
\begin{itemize}
\item \textbf{Pesquisa:} Levantamento e pesquisa acerca das propriedades matemáticas da curva elíptica e potenciais algoritmos que desafiam sua segurança.
\item \textbf{Levantamento Bibliográfico:} Levantamento bibliográfico sobre os aspectos teóricos e práticos que envolvem a segurança das curvas elípticas.
\item \textbf{Implementação:} Implementação do algoritmo Pollard-rho e suas variações.
\item \textbf{Execução:} Execução e ajustes dos algoritmos, assim como os testes de suas execuções.
\item \textbf{Parte escrita:} Escrita da parte textual do trabalho.
\end{itemize}

%
% FERRAMENTAS
%
\section{Ferramentas}
A seguir serão apresentadas as ferramentas utilizadas e o ambiente de desenvolvimento, desde as configurações físicas do computador ao software utilizado.

As especificações técnicas do computador utilizado no desenvolvimento são descritas na Tabela \ref{table:config}, a listagem dos software empregados no ambiente de desenvolvimento são descritas na Tabela \ref{table:tools} e, por fim, as bibliotecas e ferramentas de apoio são descritas na Tabela \ref{table:libs}.

\begin{table}[!ht]
\centering
\begin{tabular}{ll}
\toprule
\textbf{Item}        & \textbf{Especificação}                        \\ \midrule
Descrição            & Servidor 64 bits                              \\
\rowcolor[gray]{0.9}
Processador          & Intel(R) Xeon(R) CPU E5-2630 v2 @ 2.60GHz     \\
Memória              & 16 GiB SODIMM DDR3 Síncrono 1600 MHz (0,6 ns) \\
\rowcolor[gray]{0.9}
Armazenamento        & ATA Disk 100 MB                               \\
\bottomrule
\end{tabular}
\caption{Configuração de hardware do computador de desenvolvimento.}
\label{table:config}
\end{table}

%
% ------
%
\begin{table}[!ht]
\centering
    \begin{tabularx}{0.95\textwidth}{llcX}
    \toprule
        \textbf{Ferramenta} & {\textbf{Nome}} & \textbf{Versão} & \textbf{Comentário}  \\
    \midrule
        Sistema Operacional  & Debian Linux      & 8.3       & -                                                            \\
        \rowcolor[gray]{0.9}
        Linguagem            & C++               & 11        & O padrão C++11 foi escolhido por possuir mecanismos na própria linguagem que facilitam o desenvolvimento. \\
        Compilador           & GCC               & 3.6.1     & A saída de compilação é melhor estruturada e é compatível com os argumentos das ferramentas GNU.          \\
        \rowcolor[gray]{0.9}
        Linguagem            & Python            & 2.7       & Essa lingaugem e versão foram escolhidas por terem suporte à operações com números grandes e à biblioteca SageMath. \\
        Builder              & GNU Make          & 4.1       & -                                                                                                         \\
        \rowcolor[gray]{0.9}
        Editores de Texto    & VIM, Sublime Text & 7.4, 3114 & -                                                            \\
    \bottomrule

    \end{tabularx}
\caption{Ferramentas de desenvolvimento empregadas no processo de experimentação.}
\label{table:tools}

\end{table}

%
%
%
\begin{table}[!ht]
\centering
    \begin{tabularx}{0.95\textwidth}{lcX}

    \toprule
        \textbf{Nome}  &  \textbf{Versão}  &  \textbf{Comentário}  \\
    \midrule
        SageMath & 7.2 & Biblioteca matemática escrita em Python que provê operações para curvas elípticas.  \\
        \rowcolor[gray]{0.9}
        smtplib  & 2.6  & Biblioteca para lidar com envio de e-mails                                         \\
    \bottomrule

    \end{tabularx}
\caption{Bibliotecas e ferramentas utilizadas no desenvolvimento}
\label{table:libs}

\end{table}

%
% GERAÇÃO DE CURVAS
%
\section{Geração de Curvas}
Não é propósito deste trabalho realizar a geração de curvas elípticas e o cálculo da ordem dessas curvas. Para isso foi utilizado a biblioteca SageMath, que supre essas necessidades. Esta é uma biblioteca matemática open-source escrita em Python por uma comunidade de programadores e matemáticos, que buscam uma alternativa para os principais sistemas proprietários de software matemático como Magma, Maple, Mathematica e Matlab. SageMath é um projeto que fornece vários conceitos como teoria dos números, teoria dos grafos, combinatória, álgebra, criptografia, matemática aplicada, estatística, cálculo simbólico, etc. Para curvas elíticas, esta biblioteca implementa um rápido algoritmo de contagem de pontos (algoritmo Schoof-Elkies-Atkin), o que é bem útil na realização deste trabalho.
