\chapter{Metodologia}
Metodologias específicas colaboram para estabelecer diretrizes e boas práticas na condução do trabalho, conferindo padronização, noções de pesquisa científica, dentre ou tras contribuições. \cite{Wohlin:2000}

Com o intuito de guiar a pesquisa de forma adequada, esse capítulo aborda sobre os diversos tipos de metodologias de pesquisa, de modo a definir qual se adequa melhor ao projeto.


%
% CLASSIFICAÇÃO DA PESQUISA
%
\section{Classificação da pesquisa}
A seguir serão apresentados os grupos de classificação de pesquisa, quanto à natureza da pesquisa, abordagem do problema, objetivos de uma pesquisa e procedimentos técnicos.

\begin{itemize}
\item Natureza da pesquisa:
	\begin{itemize}
	\item \textbf{Pesquisa básica:} Possui o objetivo de gerar novos conhecimentos para ciência. Neste tipo de pesquisa não é obrigatório que o conhecimento gere um uso prático \cite{Tafner:2007}.
	\item \textbf{Pesquisa Aplicada:} Visa gerar uma maior compreensão para assuntos práticos dirigidos à solução de problemas específicos \cite{Tafner:2007}.
	\end{itemize}
\item Abordagem do problema:
	\begin{itemize}
	\item \textbf{Pesquisa Quantitativa:} O estudo quantitativo considera que tudo pode ser quantificável, ou seja, o estudo que pode ser traduzido em númerose requer uso de técnicas estatístivas para sua análise \cite{Travassos:2002}.
	\item \textbf{Pesquisa Qualitativa:} O estudo qualitativo é descritivo, sendo assim, nao pode ser analisada de forma mensurável e sim indutivamente \cite{Travassos:2002}.
	\end{itemize}
\item Objetivos de uma pesquisa:
	\begin{itemize}
	\item \textbf{Pesquisa Exploratória:} é utilizada pelo pesquisador para se familiarizar com um assunto pouco explorado. Ao decorrer ou no final da pesquisa exploratória, o pesquisador poderá estar apto para formular hipóteses \cite{Gil:2008}.
	\item \textbf{Pesquisa Descritiva:} é usada quando se tem um conhecimento do assunto e se quer descrever um fenômeno. Hipóteses podem ser formuladas com base em conhecimentos prévios, procurando confirmá-las ou negá-las \cite{Gil:2008}.
	\item \textbf{Pesquisa Explicativa:} é classificada com base em procedimentos técnicos, podendo ser quantitativa ou qualitativa. A pesquisa quantitativa traduz em números os estudos realizados e se utiliza técnicas estatísticas para comprovar os fatos \cite{Gil:2008}.
	\end{itemize}
\item Procedimentos técnicos:
	\begin{itemize}
	\item \textbf{Pesquisa bibliográfica:} é realizada a partir do levantamento de referências teóricas sobre o tema. De forma geral, toda pesquisa inicia-se como uma pesquisa bibliográfica, mas há aquelas que dependem exclusivamente desse tipo de pesquisa. Em essência, a conclusão desse tipo de pesquisa é uma compilação das publicações referentes ao tema \cite{Tafner:2007}.
	\item \textbf{Pesquisa documental:} semelhante à pesquisa bibliográfica, diferencia-se pela natureza das fontes. Tem como base documentos sem tratamento analítico, como tabelas, cartas, fotos, pinturas, dentre outros \cite{Gil:2008}.
	\item \textbf{Pesquisa experimental:} seleciona grupos de assuntos coincidentes e submete-os a tratamentos diferentes, com isso, ocorre uma verificação se há variáveis estranhas e checa-se as diferenças observadas nas respostas e se são estatisticamente significantes \cite{Tafner:2007}.
	\item \textbf{Pesquisa \textit{ex-post facto}:} a tradução literal é ``de um fato passado'', ou seja, a pesquisa \textit{ex-post facto} é realizada após a ocorrência de variáveis no objeto de estudo. A principal característica deste tipo de pesquisa é o fato de os dados serem coletados após a ocorrência dos eventos. Assim, investiga possíveis relações de causa e efeito entre um determinado fato identificado pelo pesquisador e um fenômeno que ocorre postreriormente \cite{Gil:2008}.
	\item \textbf{Levantamento (\textit{survey}):} é uma investigação realizada em retrospecto, que em seguida, mediante análise quantitativa, chega às conclusões correspondentes aos dados coletados. O levantamento feito com informações de todos os integrantes do universo da pesquisa origina um censo \cite{Travassos:2002}.
	\item \textbf{Estudo de caso:} o estudo de caso busca o aprofundamento nas questões propostas, estudando um único grupo ou comunidade, utilizando mais a observação direta do que a interrogação, captando as explicações e interpretações do que ocorre no grupo. No estudo de campo, o pesquisador está inserido no grupo, para poder entender melhor as regras, os costumes e as convenções \cite{Travassos:2002}.
	\end{itemize}
\end{itemize}

%
% METODOLOGIA DE PESQUISA
%
\section{Metodologia de pesquisa}
Analisando-se o tema proposto, observa-se uma natureza de pesquisa aplicada. Este trabalho de conclusão de curso visa a implementação do algoritmo Pollard-rho e suas variações, carregando consigo diversos conhecimentos de assuntos práticos e gerando uma solução para um problema específico, no caso, o problema do logaritmo discreto no contexto de curvas elípticas.

Considerando os objetivos de estudo deste trabalho, foi incorporada uma abordagem explicativa afim de apresentar valores quantitativos que demonstrem a relação entre os algoritmos implementados e suas diferenças, além de uma análise dos dados e resultados utilizando de técnicas com o intuito de comprovar fatos inerentes à segurança de curvas elípticas.

%
% ATIVIDADE DE PESQUISA
%
\section{Atividade de pesquisa}


O cronograma de execução deste trabalho é apresentado na Tabela \ref{cronograma tcc1}. As atividades realizadas foram:

