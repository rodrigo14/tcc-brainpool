\begin{apendicesenv}

\partapendices

\chapter{Otimizando a multiplicação de um ponto}

A multiplicação de um ponto da curva elíptica é de extrema importância para a criptografia de curvas elípticas, pois, como foi visto, sua segurança depende da dificuldade em resolver o Problema do Logaritmo Discreto para Curvas Elípticas. Como a multiplicação de um ponto é definida em termos de sucessivas somas desse ponto com ele mesmo, então a multiplicação de um ponto se torna onerosa ao multiplicá-lo por um número muito grande. Para exemplificar, considere uma curva $E/K$ e o ponto $P \in E(K)$, para calcular a multiplicação desse ponto por um número $n$ seria necessário fazer uma soma recursiva para calcular os pontos
$$
P,\quad [2]P = P + P,\quad [3]P = [2]P + P,\quad \dots,\quad [n]P = [n-1]P + P
$$
dessa forma, seria necessário realizar $n-1$ somas para encontrar o valor de $[n]P$. \cite{Silverman:2009}

Um algoritmo mais eficiente para multiplicar um ponto $P \in E(K)$ por um número $n$ é descrito abaixo (retirado de \cite{Silverman:2009})

\begin{enumerate}
  \item Escreva $n$ como uma expansão binária $$n = \epsilon_0 + \epsilon_1 \cdot 2 + \epsilon_2 \cdot 2^2 + \dots + \epsilon_t \cdot 2^t$$
com $\epsilon_0, \dots, \epsilon_t \in \{0, 1\}$ e $\epsilon_t = 1$.
  \item Atribua $Q = P$ e
	$
	R = \begin{cases}
	O, &\mbox{se} \quad\epsilon_0 = 0\\
	P, &\mbox{se} \quad\epsilon_0 = 1\\
	\end{cases}
	$
  \item Repita para $i = 1, 2, \dots, t$
    \subitem Atribua $Q = [2]Q$
    \subitem Se $\epsilon_i = 1$ então $R = R + Q$
  \item Retornar $R$
\end{enumerate}

\chapter{Algoritmo de Schoof}

O algoritmo de Schoof é um algoritmo para calcular a ordem de uma curva elíptica em um tempo polinômial, que é dado por $O(\mbox{(log \textit{n})}^8)$ \cite{Silverman:2009}.

Considere uma curva elíptica definida sobre um corpo finito $E/F_q$ e expressa pela equação $y^2 = x^3 + Ax + B$. Pelo teorema de Hasse é dado que
$$
\#E(F_q) = q + 1 - a, \qquad \mbox{com |a|} \leq 2 \sqrt{q}
$$
e considere um conjunto de números primos $S = \{2,3,5,7,\dots,L\}$ tal que 
$$
\prod_{l \in S} l \geq 4 \sqrt{q}
$$
O algoritmo consiste em encontrar $a$ mod $l$ para cada primo $l \in S$, e enfim descobrir o valor de $a$ utilizando o teorema do resto chinês.
\end{apendicesenv}
