\begin{apendicesenv}

\partapendices

\chapter{Otimizando a multiplicação de um ponto}

A multiplicação de um ponto da curva elíptica é de extrema importância para a criptografia de curvas elípticas, pois, como foi visto, sua segurança depende da dificuldade em resolver o Problema do Logaritmo Discreto para Curvas Elípticas. Como a multiplicação de um ponto é definida em termos de sucessivas somas desse ponto com ele mesmo, então a multiplicação de um ponto se torna onerosa ao multiplicá-lo por um número muito grande. Para exemplificar, considere uma curva $E/K$ e o ponto $P \in E(K)$, para calcular a multiplicação desse ponto por um número $n$ seria necessário fazer uma soma recursiva para calcular os pontos
$$
P,\quad [2]P = P + P,\quad [3]P = [2]P + P,\quad \dots,\quad [n]P = [n-1]P + P
$$
dessa forma, seria necessário realizar $n-1$ somas para encontrar o valor de $[n]P$. \cite{Silverman:2009}

Um algoritmo mais eficiente para multiplicar um ponto $P \in E(K)$ por um número $n$ é descrito abaixo (retirado de \cite{Silverman:2009})

\begin{enumerate}
  \item Escreva $n$ como uma expansão binária $$n = \epsilon_0 + \epsilon_1 \cdot 2 + \epsilon_2 \cdot 2^2 + \dots + \epsilon_t \cdot 2^t$$
com $\epsilon_0, \dots, \epsilon_t \in \{0, 1\}$ e $\epsilon_t = 1$.
  \item Atribua $Q = P$ e
	$
	R = \begin{cases}
	O, &\mbox{se} \quad\epsilon_0 = 0\\
	P, &\mbox{se} \quad\epsilon_0 = 1\\
	\end{cases}
	$
  \item Repita para $i = 1, 2, \dots, t$
    \subitem Atribua $Q = [2]Q$
    \subitem Se $\epsilon_i = 1$ então $R = R + Q$
  \item Retornar $R$
\end{enumerate}

\chapter{Algoritmo de Schoof} \label{SchoofAlg}

O algoritmo de Schoof é um algoritmo para calcular a ordem de uma curva elíptica em um tempo polinômial, que é dado por $O(\mbox{(log \textit{n})}^8)$ \cite{Silverman:2009}.

Considere uma curva elíptica definida sobre um corpo finito $E/F_q$ e expressa pela equação $y^2 = x^3 + Ax + B$. Considere ainda que $q = p^n$ e que $p \neq 2,3$. Pelo teorema de Hasse é dado que
$$
\#E(F_q) = q + 1 - t, \qquad \mbox{com |a|} \leq 2 \sqrt{q}
$$
e considere um conjunto de números primos $S = \{2,3,5,7,\dots,L\}$ tal que
$$
\prod_{l \in S} (l \geq 4 \sqrt{q})
$$
e a característica $p$ do corpo finito não pertença a $S$.
O algoritmo consiste em encontrar $t$ mod $l$ para cada primo $l \in S$, e enfim descobrir o valor de $t$ utilizando o teorema do resto chinês.\cite{Alvarado:2005}

\subsubsection{Caso com $l=2$}
Para o caso em que $l = 2$, suponha que a equação $x^3 + Ax + B$ tenha uma raiz $\alpha \in \mathbb{F}_q$, então existe um ponto $(\alpha, 0) \in E(\mathbb{F}_q)$. A soma desse ponto com ele mesmo é o ponto no infinito $\mathcal{O}$, logo esse ponto possui ordem dois e dizemos que ele pertence ao grupo de pontos da curva elíptica com ordem dois, que é representado por $(\alpha,0) \in E[2]$. O teorema de Lagrange diz que a ordem do subgrupo deve dividir a ordem do grupo, logo, se existe um subgrupo de ordem dois então a ordem do grupo será par, ou seja, $\#E(\mathbb{F}_q) = q + 1 - t \equiv 0\mbox{ (mod 2)}$, e por $q$ ser ímpar, conclui-se que $t \equiv 0 \mbox{ (mod 2)}$. \cite{Alvarado:2005}

Dessa forma, se for provado que a equação $x^3 + Ax + B$ possui uma raíz em $\mathbb{F}_q$, então $t \equiv 0 \mbox{ (mod 2)}$, caso contrário, se a raíz não existe, então $t \equiv 1 \mbox{ (mod 2)}$.\cite{McGee:2006}

Uma forma verificar se a equação $x^3 + Ax + B$ possui uma raíz em $\mathbb{F}_q$ é fazendo o cálculo do seu máximo divisor comum com o polinômio $x^q - x$. Não será demonstrado nesse trabalho que o polinômio $x^q - x$ possui $q$ raízes distintas em $\mathbb{F}_q$, logo calculando-se $\mbox{mdc}(x^3 + Ax + B, x^q - x)$ será possível saber se a equação possui uma raiz em $\mathbb{F}_q$. Caso o mdc entre os polinômios seja igual a um, então a equação $x^3 + Ax + B$ não possui raiz em $\mathbb{F}_q$ e consequentemente $t \equiv 1\mbox{(mod 2)}$.\cite{McGee:2006}

\subsubsection{Caso com $l>2$}
Para calcular $t \mbox{ mod } l$ para $l > 2$, é necessário fazer uso de uma matématica mais rebuscada. Para o propósito desse trabalho, serão dadas algumas definições e o algoritmo para realizar o calculo.

Dada uma curva elíptica $E$, o seu \textbf{j-invariante} é dado por
\begin{equation}
j = j(E) = 1728 \frac{4A^3}{4A^3 + 27B^2}\label{eq:j_inv}
\end{equation}
onde $A^3 + 27B^2 \neq 0$. \cite{Mandy:2007}

Outro conceito importante para o algoritmo de Schoof é o de polinômios de divisão. Tais polinômios são definidos sobre as coordenadas de um ponto da curva elíptica e seu resultado é zero para pontos de determinada ordem. Explicando melhor, define-se $E[n]$ como o conjunto dos pontos de $E$ que possuem ordem $n$, a saber
$$
E[n] = \{P \in E(\mathbb{F}_q)\quad |\quad nP = \mathcal{O}\}
$$
Assim, o polinômio de divisão $\psi_n$ de uma curva $E$ possui a propriedade $\psi_n(x,y) = 0$ se e somente se o ponto com coordenadas $(x,y) \in E(\mathbb{F}_q$ possui ordem $n$, ou seja, $(x,y) \in E[n]$.\cite{McGee:2006}

Esses polinômios $\psi_m(x,y)$ podem ser obtidos recursivamente por
\begin{flalign*}
& \psi_0 = 0 \\
& \psi_1 = 1 \\
& \psi_2 = 2y \\
& \psi_3 = 3x^4 + 6Ax^2 + 12Bx - A^2 \\
& \psi_4 = 4y(x^6 + 5Ax^4 + 20Bx^3 - 5A^2x^2 - 4ABx - 8B^2 - A^3) \\
& \psi_{2m+1} = \psi_{m+2}\psi_m^3 - \psi_{m-1}\psi_{m+1}^3 \\
& \psi_{2m} = \frac{\psi_m(\psi_{m+2}\psi_{m-1}^2 - \psi_{m-2}\psi_{m+1}^2)}{2y}
\end{flalign*}
com $m > 2$. \cite{Mandy:2007}

Agora, o algoritmo\cite{Mandy:2007} para calcular $t \mbox{ mod } l$ para $l > 2$
\begin{enumerate}
\item Calcular $p_l \equiv p\mbox{ mod } l$ com $|p_l| < l/2$.
\item Calcular a coordenada $x'$ de
$$(x',y') = (x^{p^2}, y^{p^2}) + p_l(x,y) \mbox{ mod } \psi_l$$
\item Para cada $h = 1,2,\cdots,\frac{l-1}{2}$, calcular a coordenada $x_h$ de $(x_h, y_h) = j(x,y)$
  \begin{enumerate}[label=(\alph*)]
  \item Se $x' - x_h^p \equiv 0 (\mbox{mod }\psi_l)$, ir para (b). Senão, tentar o próximo valor de $h$.
  \item Calcular $y'$ e $y_h$. Caso $$\frac{y'-y_h}{y} \equiv 0 (\mbox{mod }\psi_l)$$ então $t \equiv h (\mbox{mod }l)$. Caso contrário, $t \equiv -h (\mbox{mod }l)$.
  \end{enumerate}
\item Caso o item (a) da etapa anterior não encontre valor de $h$ que satisfaça a equivalência, então fazer $\omega^2 \equiv p (\mbox{mod }l)$. Caso $\omega$ não exista, então $t \equiv 0 (\mbox{mod }l)$.
\item Se $\mbox{mdc}(\mbox{numerador}(x^p - x_\omega), \psi_l) = 1$, então $t \equiv 0 (\mbox{mod }l)$. Senão, calcular $$\mbox{mdc}(\mbox{numerador}(\frac{y^p - y_\omega}{y}), \psi_l)$$
Se mdc $\neq 1$, então $t \equiv 2\omega (\mbox{mod }l)$, caso contrário $a \equiv -2\omega (\mbox{mod }l)$.
\end{enumerate}
\end{apendicesenv}
