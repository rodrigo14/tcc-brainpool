\chapter*[Introdução]{Introdução}
\addcontentsline{toc}{chapter}{Introdução}

\section*{Contextualização}
% Criptografia e suas necessidades
Criptografia é a ciência que trata de cifrar a escrita, de modo a torná-la ininteligível para os que não tenham os métodos convencionados para ter acesso a ela. Em Tecnologia da Informação, esta definição é ampliada a fim de englobar não só a escrita, mas qualquer tipo de documento ou dados que devam ser tratados secretamente. Um Sistema de Criptografia define um sistema em que um texto (ou documento, ou qualquer tipo de dado) é transformado através da criptografia em um texto cifrado (ciphertext) ou o texto cifrado é transformado de volta à informação original, através de algoritmos. A primeira ação é denominada cifragem ou criptografar, e a segunda é chamada de decifração ou decriptação. \cite{Portnoi:2005} 

% Curva elíptica e vantagens sobre RSA
A maior parte dos produtos que utilizam a criptografia de chave pública para criptografia e assinaturas digitais utiliza RSA. O tamanho da chave para o uso seguro do RSA tem aumentado nos últimos anos, e isso gerou uma carga de processamento maior sobre as aplicações usando RSA. Esse peso tem ramificações, especialmente para sites de comércio eletrônico, que realizam grandes quantidades de transações seguras. Recentemente, um sistema concorrente começou a desafiar o RSA; criptografia de curva elítica (ECC - Elliptic Curve Cryptography). O ECC já está aparecendo em esforços de padronização, como o IEEE P1363 Standard for Public-Key Cryptography. \cite{Lee:2011}

O atrativo principal do ECC, em comparação com o RSA, é que ele parece oferecer igual segurança com um tamanho de chave muito menor reduzindo assim o overhead do processamento. Por outro lado, embora a teoria do ECC já exista há algum tempo, só recentemente esses produtos começaram a aparecer, e tem havido interesse criptoanalítico contínuo em encontrar algum ponto fraco. Por conseguinte, o nível de confiança no ECC ainda não é tão alto quanto do RSA. \cite{Stallings:2011}

\section*{Problema de Pesquisa}
% Comparação entre as curvas do NIST e Brainpool

\section*{Justificativa}

\section*{Objetivos}

\begin{enumerate}
	\item \textbf{Objetivo Geral}:

	\item \textbf{Objetivos Específicos}
	\begin{enumerate}
		\item 
	\end{enumerate}
\end{enumerate}

\section*{Metodologia}
