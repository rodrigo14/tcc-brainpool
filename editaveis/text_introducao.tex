\chapter*[Introdução]{Introdução}
\addcontentsline{toc}{chapter}{Introdução}

\section*{Contextualização}
% Curva elíptica e vantagens sobre RSA
A maior parte dos produtos que utilizam a criptografia de chave pública para criptografia e assinaturas digitais utiliza RSA. O tamanho da chave para o uso seguro do RSA tem aumentado nos últimos anos, e isso gerou uma carga de processamento maior sobre as aplicações usando RSA. Esse peso tem ramificações, especialmente para sites de comércio eletrônico, que realizam grandes quantidades de transações seguras. Recentemente, um sistema concorrente começou a desafiar o RSA; criptografia de curva elíptica (ECC - \textit{Elliptic Curve Cryptography}). O ECC já está aparecendo em esforços de padronização, como o IEEE P1363 \textit{Standard for Public-Key Cryptography}. \cite{Lee:2011}

O atrativo principal do ECC, em comparação com o RSA, é que ele parece oferecer igual segurança com um tamanho de chave muito menor reduzindo assim o overhead do processamento. \cite{Stallings:2011} Em 2003, a principal empresa ligada à ECC (CERTICOM) promoveu um teste para verificação de segurança do criptossistema baseado em curvas elípticas, que foi atacado por 10.000 computadores do tipo Pentium durante 540 dias seguidos. Nesse episódio foi quebrado um sistema com chave de 109 bits, utilizando um ataque conhecido como ataque aniversário (\textit{birthday attack}). Atualmente, o tamanho mínimo de chave recomendado pelo NIST para se obter um bom nível de segurança em ECC é de, pelo menos, 163 bits; enquanto para RSA este tamanho é de 2048 bits. \cite{Sangalli:2011}

Por outro lado, embora a teoria do ECC já exista há algum tempo, só nos últimos 20 anos esses produtos começaram a aparecer, e tem havido interesse criptoanalítico contínuo em encontrar algum ponto fraco. Por conseguinte, o nível de confiança no ECC ainda não é tão alto quanto do RSA. \cite{Stallings:2011} Isso se deve ao fato de se definir alguns parâmetros antes de executar o algoritmo de cifragem/decifragem, sendo estes parâmetros não comuns em outros criptossistemas. Devemos definir inicialmente um corpo finito e, em seguida, definir a curva elíptica para que possamos gerar o grupo elíptico sobre o qual as operações serão definidas. \cite{Sangalli:2011}

\section*{Objetivos}
\begin{enumerate}
	\item \textbf{Objetivo Geral}

Objetivo deste trabalho consiste em realizar pesquisas acerca da criptografia de curvas elípticas e implementar o algoritmo Pollard-rho - que se propõe a resolver o problema matemático no qual está baseada a segurança deste sistema criptográfico - e suas variações, comparando o tempo de execução destas variações entre si.

	\item \textbf{Objetivos Específicos}
	\begin{enumerate}
		\item Apresentar conceitos matemáticos importantes sobre curvas elípticas e sua aplicação na criptografia.
		\item Demonstrar o problema do logaritmo discreto e algoritmos para resolvê-lo no contexto de ECC.
		\item Implementar um algoritmo de força bruta para resolver o problema do logaritmo discreto para curvas elípticas.
		\item Implementar o algoritmo Pollard-rho e suas variações: com único processador, com múltiplos processadores e utilizando um automorfismo como função de iteração para o Pollard-rho com múltiplos processadores.
		\item Utilizar os algoritmos implementados para solucionar o problema do logaritmo discreto para curvas elípticas mais simples, que não tenham utilização prática para criptografia de curvas elípticas. Serão utilizadas curvas elípticas mais simples por dois motivos:
		\begin{itemize}
			\item Para validar o trabalho em tempo hábil, ou seja, deseja-se obter um resultado verificável em um prazo que esteja de acordo com o calendário acadêmico da UnB para o primeiro semestre de 2016. Pois a literatura pesquisada faz referências sobre o tempo necessário para resolver o problema do logaritmo discreto para curvas de interesse criptográfico, sendo este impraticável para o escopo deste trabalho.
			\item A literatura pesquisada cita a dificuldade e o poder computacional necessário para solucionar tal problema com curvas muito complexas, fazendo alusão inclusive à experimentos já realizados utilizando-se milhares de computadores. Como não dispomos do mesmo poder computacional, optamos por resolver o problema para curvas mais simples.
		\end{itemize}
		\item Apresentar as comparações de resultados práticos dos desempenhos dos algoritmos implementados.
	\end{enumerate}
\end{enumerate}
