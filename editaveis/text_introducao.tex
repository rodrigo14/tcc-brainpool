\chapter*[Introdução]{Introdução}
\addcontentsline{toc}{chapter}{Introdução}

\section*{Contextualização}
% Curva elíptica e vantagens sobre RSA
A maior parte dos produtos que utilizam a criptografia de chave pública para criptografia e assinaturas digitais utiliza RSA. O tamanho da chave para o uso seguro do RSA tem aumentado nos últimos anos, e isso gerou uma carga de processamento maior sobre as aplicações usando RSA. Esse peso tem ramificações, especialmente para sites de comércio eletrônico, que realizam grandes quantidades de transações seguras. Recentemente, um sistema concorrente começou a desafiar o RSA; criptografia de curva elíptica (ECC - \textit{Elliptic Curve Cryptography}). O ECC já está aparecendo em esforços de padronização, como o IEEE P1363 \textit{Standard for Public-Key Cryptography}. \cite{Lee:2011}

O atrativo principal do ECC, em comparação com o RSA, é que ele parece oferecer igual segurança com um tamanho de chave muito menor reduzindo assim o overhead do processamento. \cite{Stallings:2011} Em 2003, a principal empresa ligada à ECC (CERTICOM) promoveu um teste para verificação de segurança do criptossistema baseado em curvas elípticas, que foi atacado por 10.000 computadores do tipo Pentium durante 540 dias seguidos. Nesse episódio foi quebrado um sistema com chave de 109 bits, utilizando um ataque conhecido como ataque aniversário (\textit{birthday attack}). Atualmente, o tamanho mínimo de chave recomendado pelo NIST para se obter um bom nível de segurança em ECC é de, pelo menos, 163 bits. \cite{Sangalli:2011}

Por outro lado, embora a teoria do ECC já exista há algum tempo, só recentemente esses produtos começaram a aparecer, e tem havido interesse criptoanalítico contínuo em encontrar algum ponto fraco. Por conseguinte, o nível de confiança no ECC ainda não é tão alto quanto do RSA. \cite{Stallings:2011} Isso se deve ao fato de se definir alguns parâmetros antes de executar o algoritmo de cifragem/decifragem, sendo estes parâmetros não comuns em outros criptossistemas. Devemos definir inicialmente um corpo finito e, em seguida, definir a curva elíptica para que possamos gerar o grupo elíptico sobre o qual as operações serão definidas. \cite{Sangalli:2011}

\section*{Objetivos}
\begin{enumerate}
	\item \textbf{Objetivo Geral}

Objetivo deste trabalho consiste em realizar pesquisas e testes utilizando de algoritmos de ataque que coloquem uma cuva elíptica à prova, evidenciando assim sua vulnerabilidade ou sua eficiência.

	\item \textbf{Objetivos Específicos}
	\begin{enumerate}
		\item Apresentar conceitos matemáticos importantes por trás do funcionamento de curvas elípticas e da criptografia de curvas elípticas.
		\item Demonstrar o problema do logaritmo discreto e algoritmos que tentam resolvê-lo no contexto de ECC.
		\item Implementar esses algoritmos e aplicá-los à uma curva elíptica gerada com algoritmo de geração de parâmetros.
		\item Apresentar a eficiência prática de uma curva elíptica e também dos algoritmos que o tentam quebrar através do problema do logaritmo discreto.
	\end{enumerate}
\end{enumerate}
