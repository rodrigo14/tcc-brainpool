\chapter{Considerações finais}
Nesta primeira etapa do trabalho, foram demonstrados os principais pontos quanto ao tema proposto. Em um contexto mais amplo, esta visão abrange aspectos matemáticos e conceitos relacionados à criptografia e a segurança de curvas elípticas, no qual sua característica mais marcante é a dificuldade de se resolver o problema do logaritmo discreto. Em um contexto mais técnico, foi discutida também a implementação dos algoritmos Pollard-rho e suas variações, no que se propõe a resolver o ECDLP definidas sobre corpos finitos.

Espera-se, como resultado deste trabalho, demonstrar a efetividade do algoritmo Pollard-rho no contexto descrito anteriormente. Além disso, é esperado que a versão paralelizada (com multiprocessadores) do algoritmo, seja mais rápida do que a versão original, de forma linearmente proporcional à quantidade de processadores utilizados, ou seja, apresentar valores equivalentes ao descrito na teoria, com $(\sqrt{\pi n/2})/M$ operações.

Outro fato esperado é que, com o uso de funções de automorfismo como função de iteração do algoritmo paralelizado, o tempo de execução e o número de passos necessários para solucionar o problema são reduzidos, conforme indica a literatura referenciada em \ref{eq:execParallelized}.
