\chapter{Considerações finais}
%Nesta primeira etapa do trabalho, foram demonstrados os principais pontos quanto ao tema proposto. Em um contexto mais amplo, esta visão abrange aspectos matemáticos e conceitos relacionados à criptografia e a segurança de curvas elípticas, no qual sua característica mais marcante é a dificuldade de se resolver o problema do logaritmo discreto. Em um contexto mais técnico, foi discutida também a implementação dos algoritmos Pollard-rho e suas variações, no que se propõe a resolver o ECDLP definidas sobre corpos finitos.
%
%Espera-se, como resultado deste trabalho, demonstrar a efetividade do algoritmo Pollard-rho no contexto descrito anteriormente. Além disso, é esperado que a versão paralelizada (com multiprocessadores) do algoritmo, seja mais rápida do que a versão original, de forma linearmente proporcional à quantidade de processadores utilizados, ou seja, apresentar valores equivalentes ao descrito na teoria, com $(\sqrt{\pi n/2})/M$ operações.
%
%Outro fato esperado é que, com o uso de funções de automorfismo como apoio à função de iteração do algoritmo paralelizado, o número de passos necessários para solucionar o problema são reduzidos e, consequentemente o tempo de execução do algoritmo será menor, conforme indica a literatura referenciada em \ref{eq:execParallelized}.

Com os experimentos realizados durante este trabalho, foi possível realizar uma comparação empírica entre três variantes do algoritmo Pollard-rho para ataques à criptografia de curvas elípticas. Assim, foi possível verificar os pontos fortes e as fragilidades que cada uma das abordagens oferece, bem como verificar na prática a dificuldade computacional para resolver o problema do logaritmo discreto.

Os resultados apresentados na tabela \ref{table:results} demonstram que utilizar múltiplos processos para a solução do problema é a abordagem que oferece um desempenho melhor caso seja levado em consideração apenas o tempo de execução do algoritmo. Já em contrapartida, essa abordagem trás uma preocupante restrição física para sua implementação, que é a quantidade de memória necessária para a realização dos experimentos.

Para amenizar esse problema, foi utilizado uma propriedade distintiva que selecionava os pontos que deveriam ser armazenados na memória, descartando todos os outros pontos que não atendiam tal propriedade. A escolha dessa propriedade distintiva oferece um \textit{trade-off} inevitável, pois, caso seja selecionada uma propriedade muito restritiva, a quantidade de pontos armazenados será significativamente menor, aumentando o ``tempo de vida'' da memória e permitindo que o algoritmo seja executado por mais tempo. Porém, ao restringir muito a quantidade de pontos armazenados, o espaço amostral que o algoritmo dispõe para encontrar uma colisão é menor, e portanto, será necessário mais processamento para encontrar uma colisão, e consequentemente, o tempo necessário para resolver o problema do logaritmo discreto será maior. Diante desse \textit{trade-off}, pode-se verificar a importância da escolha dessa propriedade distintiva, algo que necessita e merece mais pesquisas por parte da comunidade acadêmica.

Já os algoritmos Pollard-rho serial e paralelo não possuem o problema da memória que a versão com múltiplos processadores apresenta. A vantagem desses algoritmos é utilizar sempre a mesma quantidade de memória para os cálculos, ou seja, o consumo de memória será sempre o mesmo durante toda a execução do algoritmo. Porém, é possivel perceber pela tabela \ref{table:results} que esses algoritmos são mais lentos para resolver o problema do logaritmo discreto.