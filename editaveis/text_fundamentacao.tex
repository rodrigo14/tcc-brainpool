\chapter{Fundamentação Teórica}
Neste capítulo serão mostrados os conceitos matemáticos necessários para o entendimento dos objetivos propostos.

%
% ARITMÉTICA MODULAR
%
\section{Aritmética Modular}

O \textbf{teorema de Eudoxius} afirma que, dados dois inteiros \(a\) e \(b\) com \(b \neq 0\), então, ou \(a\) é múltiplo de \(b\), ou se encontra entre dois múltiplos 
consecutivos de \(b\), assim, existe um inteiro \(q\) tal que \cite{Santos:2014}

\begin{equation} \label{eudoxius}
  qb \leq a < (q + 1)b\quad\mbox{para \(b > 0\)}
\end{equation}

\begin{equation}
  qb \leq a < (q - 1)b\quad\mbox{para \(b < 0\)}
\end{equation}

O \textbf{teorema da divisão} diz que: dados dois inteiros $a, b$, com $b > 0$, ao dividir \(a\) por \(b\), obtém-se um único par de inteiros $q, r$ que obedecem ao seguinte relacionamento:
\begin{equation}
  a=qb+r \qquad 0 \leq r<b;\ q=\lfloor a/b \rfloor
\end{equation}

onde $\lfloor x \rfloor$ é o maior inteiro menor ou igual a \(x\) e $q,r$ são chamados respectivamente de \textit{quociente} e \textit{resto} ou \textit{resíduo} da divisão de \(a\) por \(b\) \cite{Santos:2014}. A demonstração será feita utilizando o teorema de Eudoxius descrito acima. Dessa forma, existe um inteiro \(q\) que satisfaz a equação~\ref{eudoxius}, o que implica que \(0 \leq a - qb\) e que \(a - qb < b\). Definindo \(r = a - qb\), teremos demonstrado a existência de \(q\) e \(r\). A unicidade de \(q\) e 
\(r\) pode ser demonstrada por contradição. Suponhamos que exista outro par de inteiros \(q_1\) e \(r_1\) que satisfaçam a equação
\begin{equation}
  a = q_1b + r_1
\end{equation}
então \((qb + r) - (q_1b - r_1) = 0\), assim temos que \(b(q - q_1) = r_1 - r\), ou seja, \(b|(r_1 - r)\). Mas isso só é possível se \(r_1 - r = 0\) 
pois \(r < b\) e \(r_1 < b\), dessa forma temos que \(r_1 = r\) 
e assim fica também demonstrada a unicidade de \(q\) e \(r\). \cite{Santos:2014}


A aritmética modular trata da relação entre os inteiros e seus resíduos quando são divididos por algum inteiro positivo denominado \textbf{módulo} \cite{Lewinter:2015}. 
Para exemplificar melhor, considere a sequência de números naturais $S = 3, 4, 5, \dots$ e o módulo sendo $b = 3$, é possível reescrever qualquer número dessa sequência 
como \(3q\), \(3q + 1\) ou \(3q + 2\). Dessa forma, qualquer número da sequência se encaixa em uma dessas três classes. Analogamente, se o módulo for \(n\), então haverá 
\(n\) \textbf{classes de resto mod n}. A aritmética modular estuda essa relação entre cada inteiro da sequência \(S\) e sua classe de resto mod \(n\) \cite{Lewinter:2015}.

Quando dois inteiros pertencem à mesma classe de resto, diz-se que estes são \textit{congruentes mod k}. Uma definição mais formal de congruência é: dados dois inteiros \(a\) e \(b\), diz-se que \(a\) é \textit{congruente} a \(b\) módulo \(m\) se \(m\) divide a diferença $(a-b)$, ou seja, $m \mid (a-b)$. A relação de congruência entre \(a\) e \(b\) é representada pela notação abaixo.
\begin{equation}
  a \equiv b \pmod m \label{eq:1}
\end{equation}

Consequentemente, se \(m\) não divide a diferença $(a - b)$, então \(a\) é \textit{incongruente} a \(b\) \textit{mod m}, e representa-se por $a \not\equiv b \pmod m$ \cite{Santos:2014}.

Seja um inteiro positivo \(n\) e \(a\), \(b\), \(c\) inteiros quaisquer, a congruência tem as seguintes propriedades \cite{Stallings:2011}:
\begin{enumerate}
  \item $a \equiv b \pmod n$ se $n|(a - b)$.
  \item $a \equiv b \pmod n$ implica $b \equiv a \pmod n$.
  \item $a \equiv b \pmod n$ e $b \equiv c \pmod n$ implica $a \equiv c \pmod n$.
\end{enumerate}

Por definição, o operador $\pmod n$ mapeia todos os inteiros para o conjunto de inteiros $\{0, 1, ..., (n-1)\}$. Daí pode-se realizar operações aritméticas dentro dos limites desse conjunto, técnica conhecida como aritmética modular. A aritmética modular exibe as seguintes propriedades \cite{Stallings:2011}:
\begin{enumerate}
  \item $[a \pmod n + b \pmod n] \pmod n = (a + b) \pmod n$
  \item $[a \pmod n - b \pmod n] \pmod n = (a - b) \pmod n$
  \item $[a \pmod n \times b \pmod n] \pmod n = (a \times b) \pmod n$
\end{enumerate}

% DESENVOLVER O MDC
Antes de tratar a divisão módulo \(n\), é importante fixar o conceito de \textbf{máximo divisor comum}(MDC) entre dois números inteiros. O MDC entre dois inteiros \(a\) e \(b\) 
é o maior inteiro que divide simultaneamente \(a\) e \(b\).

A divisão módulo $n$ não está definida em todos os casos. Seja \(d\) um número inteiro que divide \(a\) e \(b\), verifica-se que a relação apresentada na Equação \ref{eq:3}, onde mdc\((n, d)\) é o máximo divisor comum entre \(n\) e \(d\).

\begin{equation}
  \frac{a}{d} \equiv \frac{b}{d}\left(\mbox{mod}\ \frac{n}{\text{mdc}(n,d)}\right) \label{eq:3}
\end{equation}

O inverso multiplicativo de $a$ módulo \(n\) é o número inteiro \(b\) que satisfaz a operação \(a \times b = 1 \pmod  n\), e somente pode ser obtido quando existe \((a, n) = 1\) (isto é, \(a\) e \(n\) são coprimos). O valor do inverso multiplicativo \(b\) pode ser obtido através do Algoritmo de Euclides Estendido \cite{Halim:2013}.

\par Outra operação que pode ser definida é a exponenciação modular, que calcula o resto de um número inteiro \(b\) quando elevado a um número inteiro \(k\) e é dividido por um inteiro positivo \(m\). Assim, $b$ é a base, $k$ o expoente e $m$ o módulo da exponenciação modular, que pode ser expressa como na equação \ref{eq:4}.

\begin{equation}
  c \equiv b^k\pmod m \label{eq:4}
\end{equation}


%
% RELAÇÕES DE EQUIVALÊNCIA
%
\section{Relações de Equivalência}

Em matemática, relação é o estudo de como os elementos de um conjunto $E$, ou de dois conjuntos distintos, se relacionam entre si \cite{Domingues:2003}. O conceito de relação pode ser compreendido facilmente partindo-se da definição de outro conceito, o de \textit{produto cartesiano}.

Sejam dois conjuntos não vazios \(E\) e \(F\), o \textbf{produto cartesiano} entre esses dois conjuntos, representado por \(E \times F\), é o conjunto de todos os pares ordenados \((x, y)\), tais que \(x \in E\) e \(y \in F\).

\begin{equation}
  E \times F = {(x, y) | x \in E \mbox{ e } y \in F}
\end{equation}

Sabendo-se o conceito de produto cartesiano, defini-se \textit{relação binária} de \(E\) em \(F\) como qualquer subconjunto de \(E \times F\), ou seja, uma relação \(\mathcal{R}\) une um \(x \in E\) a um \(y \in F\) em um par ordenado \((x, y)\), e dizemos que \textit{\(x\) se relaciona com \(y\) mediante \(\mathcal{R}\)}, que pode ser representado por \(x\mathcal{R}y\) \cite{Domingues:2003}. Lembrando-se que, uma relação \(\mathcal{R}\) pode ser definida somente sobre um conjunto \(E\), ou seja, a relação \(\mathcal{R}\) é um subconjunto do produto cartesiano \(E \times E\).

Quando definida sobre um conjunto \(E\), algumas propriedades que a relação \(\mathcal{R}\) pode atender são:

\begin{itemize}
  \item \textbf{Reflexiva}: a relação \(\mathcal{R}\) é reflexiva se todo elemento \(x \in E\) se relaciona consigo mesmo, ou seja, \(x\mathcal{R}x\). Por exemplo, para \(E = \{a, b, c\}\), a relação 
  $$\mathcal{R} = \{(a, a), (b, b), (c, c), (a, b), (a, c)\}$$
  é reflexiva, pois para cada elemento \(x \in E\), existe um par ordenado \((x, x)\) em \(\mathcal{R}\) \cite{Domingues:2003}. 

  \item \textbf{Simétrica}: a relação \(\mathcal{R}\) é simétrica se é válido afirmar que \(y\mathcal{R}x\) sempre que \(x\mathcal{R}y\), com \(x, y \in E\). Por exemplo, a relação 
  $$\mathcal{R} = \{(a, a), (a, b), (b, a), (c, c)\}$$
  é simétrica sobre \(E = \{a, b, c\}\), pois é facilmente verificável que \(y\mathcal{R}x\) sempre que \(x\mathcal{R}y\) \cite{Domingues:2003}.

  \item \textbf{Transitiva}: a relação \(\mathcal{R}\) é transitiva se é válido afirmar que \(x\mathcal{R}z\) sempre que \(x\mathcal{R}y\) e \(y\mathcal{R}z\). Por exemplo, a relação 
  $$\mathcal{R} = \{(a, b), (b, c), (a, c), (c, c)\}$$
  é transitiva sobre \(E = \{a, b, c\}\), pois pode-se verificar que \(a\mathcal{R}b\) e \(b\mathcal{R}c\) implica em \(a\mathcal{R}c\) \cite{Domingues:2003}.

  \item \textbf{Anti-simétrica}: a propriedade anti-simétrica de uma relação \(\mathcal{R}\), como o nome já diz, é o contrário da propriedade simétrica, ou seja, se o par ordenado \((x, y) \in \mathcal{R}\), então \((y, x) \notin \mathcal{R}\), exceto se \(x = y\). Por exemplo, a relação 
  $$\mathcal{R} = \{(a, a), (a, b), (b, c), (c, c)\}$$
  é anti-simétrica sobre \(E = \{a, b, c\}\), pois para cada par \((x, y) \in \mathcal{R}\), com \(x \neq y\), o par \((y, x) \notin \mathcal{R}\).
\end{itemize}

Uma relação que satisfaz as propriedades reflexiva, simétrica e transitiva citadas acima, é chamada de \textbf{relação de equivalência} \cite{Domingues:2003}.

Seja o conjunto \(E\) e a relação de equivalência \(\mathcal{R}\) sobre \(E\), para cada elemento \(a \in E\) existe uma \textbf{classe de equivalência} representada por \(\overline{a}\) e constituída por todos os elementos em \(E\) que estão relacionados com \(a\) mediante a relação \(\mathcal{R}\)
$$
  \overline{a} = \{x \in E | x \mathcal{R} a \}
$$
e diz-se que \(\overline{a}\) é a classe de equivalência determinada por \(a\), módulo \(\mathcal{R}\) \cite{Domingues:2003}. Por exemplo, na relação de equivalência 
$$
\mathcal{R} = \{(a, a), (b, b), (c, c), (d, d), (a, b), (b, a), (b, c), (c, b), (a, c), (c, a)\}
$$
sobre o conjunto \(E = \{a, b, c, d\}\), a classe de equivalência determinada por \(a\) módulo \(\mathcal{R}\) é dada por
$$
  \overline{a} = \{a, b, c\}
$$
pois apenas os elementos \(a\), \(b\) e \(c\) estão relacionados com \(a\) mediante \(\mathcal{R}\).

O conjunto formado pelas classes de equivalência módulo \(\mathcal{R}\) é chamado de \textbf{conjunto-quociente} de \(E\) por \(\mathcal{R}\) e indicado por \(E/\mathcal{R}\) \cite{Domingues:2003}. Por exemplo, para o exemplo anterior, as classes de equivalência são
$$
  \overline{a} = \{a, b, c\}, \quad
  \overline{b} = \{b, a, c\}, \quad
  \overline{c} = \{c, a, b\}, \quad
  \overline{d} = \{d\}
$$
logo, o conjunto-quociente de \(E\) por \(\mathcal{R}\) é dado por
$$
  E/\mathcal{R} = \{\{a, b, c\}, \{d\}\}
$$

%
% ESTRUTURAS ALGÉBRICAS
%
\section{Estruturas Algébricas}

Antes de iniciar os estudos em aritmética de curvas elípticas, é importante
fixar alguns conceitos de álgebra. De acordo com Hefez, estruturas algébricas são modelos abstratos para tratar em bloco várias situações matemáticas concretas, onde determinados conjuntos definem operações com propriedades semelhantes \cite{Hefez:2008}.

%
% GRUPOS
%
\subsection{Grupos}

Um \textbf{grupo \((G,*)\)} é composto por um conjunto \(G\) e uma operação binária \(*\) sobre os elementos desse conjunto tal que os axiomas abaixo sejam satisfeitos \cite{Gilbert:2004}.

\begin{enumerate}
\item O conjunto \(G\) é \textbf{fechado} para a operação \(*\)

$a * b \in G$ para todo $a,b \in G$.

\item A operação $*$ é \textbf{associativa}

$(a * b) * c = a * (b * c)$ para todo $a,b,c \in G$.

\item Existe um \textbf{elemento identidade} $e \in G$ tal que para todo elemento $a \in G, a * e = a$.
\item Existe um \textbf{elemento inverso}\footnote{Para um grupo é necessário que um elemento \(a\) tenha um elemento \textit{simétrico}, denotado genericamente por \(a'\). Se a operação do grupo for a multiplicação, então o elemento simétrico é chamado de \textit{inverso} de \(a\), e é representado por \(a^{-1}\). Se a operação do grupo for a soma, então o elemento simétrico é chamado de \textit{negativo} de \(a\), e é representado por \(-a\) \cite{Domingues:2003}. Para manter a simplicidade da notação nesse texto, o simétrico sempre será equivalente ao inverso, mesmo que a operação do grupo seja a soma.} \(a^{-1}\) para todo elemento $a \in G$ tal que $a^{-1} * a = e$ (elemento identidade).
\end{enumerate}

Se a operação é \textbf{comutativa}, ou seja, se $a * b = b * a$ para todo $a, b \in G$, então o grupo é denominado \textbf{abeliano} (ou \textbf{comutativo}) em homenagem ao matemático Niels Abel \cite{Gilbert:2004}.

Exemplos de grupos abelianos são $(\mathds{Z}, +)$, $(\mathds{R}, +)$, ambos com identidade 0 e com infinitos elementos. A \textbf{ordem do grupo} é dada pela quantidade de elementos que este possui \cite{Coutinho:2014}. Assim, um grupo que possui infinitos elementos, por exemplo, o grupo $(\mathds{Z}, +)$, diz-se que possui ordem infinita, já os grupos que possuem uma quantidade finita de elementos, diz-se possuir orderm finita. Para este trabalho, os grupos de maior interesse são aqueles que possuem ordem finita.

%
% SUBGRUPOS
%
\subsection{Subgrupos}
Seja $(G, *)$ um grupo, e \(H\) um subconjunto não vazio de \(G\). Se $(H, *)$ satisfaz todos os axiomas de grupo, então diz-se que $(H, *)$ é um subgrupo de $(G, *)$ \cite{Coutinho:2014}, ou seja

\begin{enumerate}
\item Para todo $a, b \in H$, $a * b \in H$.
\item O elemento neutro de \(H\) é o mesmo elemento neutro de \(G\).
\item Existe um inverso \(a'\) para todo $a \in H$.
\end{enumerate}

Todo grupo possui pelo menos dois subgrupos. Esses dois subgrupos são chamados de \textbf{subgrupos triviais}. Um deles é o subgrupo formado unicamente pelo elemento neutro, ou seja, por \(e\), pois esse grupo, apesar de formado por um conjunto composto somente pelo elemento neutro, satisfaz todos os axiomas de grupo para a operação dada. O segundo subgrupo trivial de um grupo (\(G, *)\) é ele mesmo \cite{Domingues:2003}.

%
% TEOREMA DE LAGRANGE
%
\subsection{Teorema de Lagrange}
Um teorema de grande importância para o estudo da criptografia é o teorema de Lagrange, no qual estabelece que, se \(G\) é um grupo finito e \(H\) é um subgrupo de \(G\), então a ordem de \(H\) sempre divide a ordem de \(G\) \cite{Shoup:2005}. A demostração deste teorema não será abordada nesse trabalho, mas seu resultado será de grande utilidade mais à frente, no estudo da criptografia de curvas elípticas.

%
% CLASSES LATERAIS E DEMONSTRAÇAO DO TEOREMA DE LAGRANGE
%

%
% SUBGRUPOS CÍCLICOS
%
\subsection{Subgrupos cíclicos}
Seja o grupo finito $(G,*)$ e um elemento $a \in G$. É possível realizar a operação $*$ do grupo repetidas vezes sobre o elemento $a$, obtendo
$$
a^k = a * a * \ldots * a \quad\mbox{(k vezes)}
$$
que também é um elemento do grupo finito $(G,*)$, pois, pela definição de grupo, o conjunto \(G\) deve ser fechado para a operação \(*\). Pode-se dizer que essa é a k-ésima \textit{potência} de $a$ (assim, \(a * a = a^2\) mesmo que a operação $*$ do grupo não seja equivalente à multiplicação). 
Assim, é possível criar um conjunto com as potências de $a$, obtendo
$$
H = \{e, a, a^2, a^3, \ldots\}
$$
que é um subconjunto de $G$. Como o conjunto $G$ é um conjunto finito, então o conjunto $H$ também é finito e existem inteiros positivos $n > m$, tais que $a^m = a^n$, por exemplo, se \(m = 0\) temos que \(a^m = a^0 = e\), que é o elemento neutro, e também existe um inteiro \(n > 0\) tal que \(a^n = e\). 

Considere que o inverso de $a$ seja $a^{-1}$, multiplicando ambos os lados desta equação por $(a^{-1})^m$, obtem-se $a^{m - m} = a^{n-m} = e$, que é o elemento neutro \cite{Coutinho:2014}. Daí pode-se afirmar que
\begin{enumerate}
\item Dado um elemento $a \in G$, existe um inteiro positivo $k$ tal que $a^k = e$.
\item Se $a^k = e$, então o inverso de $a$ é dado por $a^{k-1}$, pois $a * a^{k-1} = a^k = e$. Por ser uma potência de $a$, o inverso de $a$ também pertence ao conjunto $H$.
\end{enumerate}

A \textbf{ordem do elemento} $a \in G$ é o menor inteiro positivo $k$ tal que $a^k = e$. Como o elemento $a$ gera o conjunto $H$, diz-se que $a$ é um \textbf{gerador} do conjunto $H$, e consequentemente, a ordem do conjunto $H$ é igual à ordem de $a$. Quando um grupo é formado por um elemento gerador, diz-se que tal grupo é um \textit{grupo cíclico} \cite{Coutinho:2014}.

Pelo teorema de Lagrange, é possível concluir que a ordem de $H$ divide a ordem de $G$. Uma importante consequência dessa afirmação é que, se um grupo $G$ possui ordem $p$ prima, então seus subgrupos devem ter ordem $1$ ou $p$. Porém, como um subgrupo $H$ formado pelas potências de um elemento $a$ deve possuir pelo menos o elemento neutro $e$ e um gerador $a$, ou seja, deve possuir pelo menos dois elementos, então todos os subgrupos de $G$, com exceção do subgrupo trivial formado pelo elemento neutro, devem ter ordem $p$. Em consequência disso, qualquer elemento $b \in G$ é um gerador do grupo e todo grupo de ordem prima é um grupo cíclico, embora nem todo grupo cíclico possua ordem prima. Por exemplo, o grupo formado formado pelo conjunto $G=\{1,2,3,4\} \mbox{(mod 5)}$ e a operação de multiplicação têm ordem $k = 4$, porém admite-se o elemento 2 como gerador \cite{Coutinho:2014}.

%
% HOMOMORFISMOS DE GRUPO
%
\subsection{Homomorfismos de grupos}
Sejam os grupos $(G, \cdot)$ e $(H, \cdot)$, uma função $f: G \rightarrow H$ é chamada de \textit{homomorfismo} se $f(a \cdot b) = f(a) \cdot f(b)$ onde $a, b \in G$. Se a operação dos grupos for diferente, por exemplo, $(G, \cdot)$ e $(H, \star)$, então a condição é dada por $f(a \cdot b) = f(a) \star f(b)$ \cite{Gilbert:2004}.

% Explicar o que é uma função bijetora (notas de rodapé)
Sendo $f: G \rightarrow H$ uma função de homomorfismo entre os grupos \(G\) e \(H\), se a função for bijetora, então ela é denominada \textit{isomorfismo}, e ainda, se os grupos \(G\) e \(H\) são iguais e a função é um isomorfismo, então ela é denominada \textit{automorfismo} \cite{Shokranian:2010}.

%
% CORPOS FINITOS
%
\subsection{Corpos Finitos}
Corpos finitos são estruturas algébricas compostas por um conjunto numérico \(\mathbb{F}\) juntamente com duas operações, sendo uma delas a operação de soma (denotada por $+$) e a outra a operação de multiplicação (denotada por $\cdot$) onde são satisfeitas as seguintes propriedades: \cite{Guide}

\begin{enumerate}
\item $(\mathbb{F}, +)$ é um grupo abeliano com elemento identidade sendo \(0\).
\item $(\mathbb{F} \backslash \{0\}, \cdot)$ é um grupo abeliano com elemento identidade sendo $1$.
\item Para todo $a, b, c \in \mathbb{F}$, é válida a propriedade distributiva $(a + b) \cdot c = a \cdot c + b \cdot c$.
\end{enumerate}

Se o conjunto \(\mathbb{F}\) é finito, então o corpo é denominado \textit{corpo finito}. A \textit{ordem} de um corpo finito indica a quantidade de elementos que constituem o corpo finito. Existe um corpo finito de ordem \(q\) se e somente se \(q\) é igual a uma potência prima, ou seja, $q = p^{m}$, onde \(p\) é um número primo. O número \(p\) é chamado de característica do corpo finito. Se $m = 1$ então o corpo é denominado \textit{corpo primo}, caso $m \geq 2$, o corpo é chamado de \textit{corpo extenso} \cite{Guide}.

%
% CRIPTOGRAFIA
%
\section{Criptografia} \label{sec:criptografia}
A criptografia é uma ciência que tem por objetivo estudar as formas de cifrar e decifrar a escrita, de modo a torná-la ininteligível para aqueles que não devem ter acesso à informação nela contida. Em Tecnologia da Informação, a criptografia é utilizada para cifrar não apenas a escrita, mas qualquer tipo de dados que necessitem de um tratamento mais restrito quanto ao seu acesso. Um Sistema de Criptografia define procedimentos para cifrar um texto (ou documento, ou qualquer tipo de dado), transformando-o em um texto cifrado (\textit{ciphertext}) e decifrar o mesmo texto cifrado, obtendo-se novamente o texto original (\textit{plaintext}), através de algoritmos. A primeira ação é denominada cifração e a segunda é chamada de decifração \cite{Portnoi:2005}.

A criptografia, em geral, fornece quatro serviços: \cite{Portnoi:2005}

\begin{itemize}
\item \textit{confidencialidade}: é a cifração dos dados de forma que somente seja inteligível aos receptores legítimos.
\item \textit{integridade}: durante a cifração dos dados, é necessário garantir que a informação não seja modificada pelo algoritmo de cifração, ou seja, a forma como os dados se apresentam deve ser modificada, mas sua informação deve permanecer a mesma.
\item \textit{autenticação}: é a garantia de que o emissor e o receptor são realmente quem dizem ser, ou seja, é a forma de verificar as identidades uns dos outros assim como a origem e o destino da informação. A autenticação é obtida através da assinatura digital e dos certificados.
\item \textit{irretratabilidade}: impede que o emissor ou o receptor negue a mensagem transmitida, dessa forma, o emissor pode provar que o receptor recebeu a mensagem e o receptor pode provar que o emissor a enviou.
\end{itemize}

%
% CRIPTOGRAFIA SIMÉTRICA
%
\subsection{Criptografia simétrica}
A criptografia simétrica, ou convencional, foi o primeiro tipo de criptografia utilizado. Baseia-se no uso de uma chave secreta para cifrar e decifrar a mensagem. É importante notar que a mesma chave utilizada para cifrar a mensagem é utilizada para decifrá-la. Dessa forma, é necessário que o emissor e o receptor saibam interpretar a chave secreta, assim, é necessário utilizar um canal seguro para compartilhar tal chave. Esse tipo de criptografia é bastante utilizado em transmissões de dados que exista uma forma barata de estabelecer um canal seguro, tais como como mensagens enviadas de um computador para outro, nas comunicações entre duas máquinas ou no armazenamento da informação em um disco rígido, dentre outras \cite{Cavalcante:2015}.

%
% CRIPTOGRAFIA ASSIMÉTRICA
%
\subsection{Criptografia assimétrica}
A criptografia assimétrica, também conhecida como criptografia de chave pública, utiliza duas chaves distintas, uma para cifrar os dados (chamada de chave pública) e outra para decifrar os dados (chamada de chave privada). Neste caso a chave pública é divulgada enquanto que a chave privada permanece secreta. Assim, o emissor tem acesso à chave pública do receptor e utiliza esta para cifrar a mensagem. O receptor ao receber a mensagem, utiliza sua chave privada para decifrá-la. Esse tipo de criptografia não precisa de um canal seguro para compartilhamento da chave, pois a chave pública está disponível para qualquer um que queira enviar a mensagem para o receptor, porém, somente o receptor possui a chave privada \cite{Cavalcante:2015}.
Este sistema é capaz provê a confidencialidade dos dados e também autenticação pois somente o proprietário da chave privada será capaz de decifrar a mensagem. Em algoritmos dessa natureza, um dado que é criptografado com uma chave só poderá ser decifrado com a utilização da outra chave e vice-versa.
