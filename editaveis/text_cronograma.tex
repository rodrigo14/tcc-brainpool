%
% CRONOGRAMA
%
\section{Cronograma}
O cronograma de execução deste trabalho é apresentado nas Tabelas \ref{cronograma tcc1} e \ref{cronograma tcc2}. Objetivo é conferir uma noção temporal acerca das atividades definidas. Trata-se de uma visão preliminar, logo, pode haver modificações, ao longo do projeto de acordo com as necessidades.

\begin{table}[h]
\centering
\caption{Cronograma do TCC1}
\label{cronograma tcc1}
\begin{adjustbox}{width=\textwidth}
\begin{tabular}{|l|c|c|c|c|c|c|}
\hline
                                                & \multicolumn{1}{l|}{\textbf{Agosto}} & \multicolumn{1}{l|}{\textbf{Setembro}} & \multicolumn{1}{l|}{\textbf{Outubro}} & \multicolumn{1}{l|}{\textbf{Novembro}} & \multicolumn{1}{l|}{\textbf{Dezembro}} \\ \hline
\textbf{Elaborar Proposta Inicial}              & X                                    & X                                      &                                       &                                        &                                        \\ \hline
\textbf{Definir Escopo}                         & X                                    & X                                      &                                       &                                        &                                        \\ \hline
\textbf{Definir Metodologia}                    &                                      & X                                      & X                                     &                                        &                                        \\ \hline
\textbf{Realizar Levantamento Bibliográfico}    &                                      & X                                      & X                                     & X                                      &                                        \\ \hline
\textbf{Estabelecer Proposta}                   &                                      & X                                      & X                                     & X                                      &                                        \\ \hline
\textbf{Desenvolver Prova de Conceito}          &                                      & X                                      & X                                     & X                                      &                                        \\ \hline
\textbf{Apresentar TCC1}                        &                                      &                                        &                                       &                                        & X                                      \\ \hline
\textbf{Refinar TCC1}                           &                                      &                                        &                                       &                                        & X                                      \\ \hline
\end{tabular}
\end{adjustbox}
\end{table}

\begin{table}[h]
\centering
\caption{Cronograma do TCC2}
\label{cronograma tcc2}
\begin{adjustbox}{width=\textwidth}
\begin{tabular}{|l|c|c|c|c|c|c|}
\hline
                                            & \multicolumn{1}{l|}{\textbf{Março}} & \multicolumn{1}{l|}{\textbf{Abril}} & \multicolumn{1}{l|}{\textbf{Maio}} & \multicolumn{1}{l|}{\textbf{Junho}} & \multicolumn{1}{l|}{\textbf{Julho}} \\ \hline
\textbf{Implementação PR único processador} & X                                   & X                                   & X                                  &                                     &                                     \\ \hline
\textbf{Implementação PR paralelizado}      & X                                   & X                                   & X                                  &                                     &                                     \\ \hline
\textbf{Implementação PR com automorfismo}  &                                     & X                                   & X                                  & X                                   &                                     \\ \hline
\textbf{Coletar e Avaliar os Resultados}    &                                     &                                     & X                                  & X                                   &                                     \\ \hline
\textbf{Apresentar TCC2}                    &                                     &                                     &                                    & X                                   &                                     \\ \hline
\textbf{Refinar TCC2}                       &                                     &                                     &                                    &                                     & X                                   \\ \hline
\end{tabular}
\end{adjustbox}
\end{table}
