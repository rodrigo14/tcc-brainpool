\begin{anexosenv}

\partanexos

\chapter{Primeiro Anexo} \label{alg:single}
Este anexo se refere à Seção \ref{sec:single}.

\textit{O algoritmo recebe como parâmetro uma curva elíptica \(E\) e dois pontos \(P\) e \(Q\) pertencentes a \(E\) e calcula o valor de \(x\) tal que $x = \log_P Q$.}

%
% INPUT: EllipticCurve E, Point P, Point Q
% OUTUPT: x = log_P(Q)
%
\begin{lstlisting}[caption={Algoritmo Pollard-rho com único processador.},label=single_processor]
pollardRho_singleProcessor(E: Curva, P: Ponto, Q: Ponto): inteiro
	seja L, k, inteiro
	seja an, bn, inteiro
	seja am, bm, inteiro
	seja Xn, Xm, Ponto
	seja c, d, array inteiro
	seja R, array Ponto

	k := E.order()
	para j := 1, enquanto j <= L, faca
		c[j] := random() % k
		d[j] := random() % k
		R[j] := c[j]*P + d[j]*Q

	an := random() % k
	bn := random() % k
	Xn := an*P + bn*Q
	am := an
	bm := bn
	Xm := Xn

	enquanto Xn != Xm faca
		j := H(Xn, L)
		Xn := Xn + R[j]
		an := an + c[j]
		bn := bn + b[j]

	se bn = bm entao
		retorne "falha"

	seja x, inteiro
	x := (an - am)/(bm - bn) % k
	retorne x

H(P: Ponto, L: inteiro): inteiro
	retorne P.x % L + 1

\end{lstlisting}

\chapter{Segundo Anexo} \label{alg:parallelized}
Este anexo se refere à Seção \ref{sec:parallelized}.

\textit{O algoritmo recebe como parâmetro uma curva elíptica \(E\) e dois pontos \(P\) e \(Q\) pertencentes a \(E\) e calcula o valor de \(x\) tal que $x = \log_P Q$.}

%
% INPUT: EllipticCurve E, Point P, Point Q
% OUTUPT: x = log_P(Q)
%
\begin{lstlisting}[caption={Algoritmo Pollard-rho paralelizado.},label=parallelized]
pollardRho_parallelized(E: Curva, P: Ponto, Q: Ponto): inteiro
	seja L, k, inteiro
	seja an, bn, inteiro
	seja am, bm, inteiro
	seja Xn, Xm, Ponto
	seja c, d, array inteiro
	seja R, array Ponto

	seja a, b, inteiro
	seja X, ponto
	seja Y, Point

	k := E.order()
	para j := 1, enquanto j <= L, faca
		c[j] := random() % k
		d[j] := random() % k
		R[j] := c[j]*P + d[j]*Q

	# para cada processador M faca
	a := random() % k
	b := random() % k
	X := an*P + bn*Q

	# repita ate que o servidor receba um ponto Y distinto pela 2a vez
	enquanto duplicatedDistinguishedPointNotReceived() faca
		se X = distinguishedPoint(X)
			sendToServer(a, b, X)
			j := H(X)
			X := X + R[j]
			a := a + c[j] % k
			b := b + d[j] % k

	an := firstTripleCollided_a()
	bn := firstTripleCollided_b()
	am := secondTripleCollided_a()
	bm := secondTripleCollided_b()

	se bn = bm entao
		retorne "falha"

	seja x, inteiro
	x := (an - am)/(bm - bn) % k
	retorne x

H(P: Ponto, L: inteiro): inteiro
	retorne P.x % L + 1

\end{lstlisting}

\chapter{Terceiro Anexo} \label{curve_parameters}
A seguir serão descritos os parâmetros das curvas utilizadas e os pontos selecionados na elaboração do problema do logaritmo discreto (ECDLP).

\begin{table}[h]
\centering
\caption{Parâmetros das curvas}
\label{table:required_chips}
\begin{tabular}{|c|c|c|c|c|}
\hline
\multicolumn{1}{|c|}{\textit{bits}} & \textbf{A} & \textbf{B} & \textbf{p} & \textbf{ordem} \\ \hline
32 & 48489 & 97594 & 2882746957 & 2882734229 \\ \hline
36 & 50533 & 32996 & 40682286653 & 40682161199 \\ \hline
40 & 42144 & 98649 & 950263177883 & 950264094121 \\ \hline
44 & 40198 & 11702 & 16020544557377 & 16020543189151 \\ \hline
48 & 52920 & 71257 & 228362934047953 & 228362914116653 \\ \hline
52 & 99400 & 82990 & 3504467965402487 & 3504467954199811 \\ \hline
56 & 50404 & 48342 & 51490278520613677 & 51490278431089031 \\ \hline
60 & 82106 & 75478 & 706856769607124263 & 706856770750892953 \\ \hline
64 & 98626 & 27290 & 10136088024214031251 & 10136088023118239857 \\ \hline
65 & 4838 & 85770 & 18505261939516721599 & 18505261939128715183 \\ \hline
66 & 78319 & 6465 & 53604234314297933089 & 53604234316981472713  \\ \hline
\end{tabular}
\end{table}


\begin{table}[h]
\centering
\caption{Pontos utilizados no ECLDP}
\label{table:ecldp_points}
\begin{tabular}{|p{1cm}|p{4.5cm}|p{4.5cm}|}
\hline
\multicolumn{1}{|c|}{\textit{bits}} & \textbf{P (x, y)} & \textbf{Q (x, y)} \\ \hline
36 & (25075567916, 24630560942) & (10330877254, 15721202004) \\ \hline
40 & (573167557027, 832383072150) & (630765322529, 829885145927) \\ \hline
44 & (7321690921039, 5900903534434) & (15339793358558, 5065066467746) \\ \hline
48 & (90391684716440, 216804106155420) & (189599746109552, 60036328328093) \\ \hline
52 & (730734419482594, 2275382146632833) & (601104431610521, 3428181333227735) \\ \hline
56 & (27290054680121669, 33873600851117647) & (17717646814745382, 41475147684384198) \\ \hline
60 & (279391159031135790, 412346973450156951) & (477223543628388853, 375727204622926174) \\ \hline
64 & (2310651219632187277, 8645536242470898633) & (4494019437538409888, 5769703912640477331) \\ \hline
65 & (7641240624886147510, 16388105095474639144) & (15127239423762891439, 12518025431655074976) \\ \hline
66 & (16503064980171743392, 44363827824736096237) & (38223962489371357849, 9876943738967192018) \\ \hline
\end{tabular}
\end{table}

\end{anexosenv}
