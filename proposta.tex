\input{fixos/pacotes}
\input{fixos/comandos}
\input{fixos/novosComandos}

% Dados pessoais
\autor{Rodrigo Santana Gonçalves e Carlos Alberto Teixeira Junior}
\curso{Engenharia de Software}

% Dados do trabalho
\titulo{Estudo da Segurança Criptográfica De Algoritmos De Curva Elíptica}
\data{2015}
\palavraChaveUm{Palavra-chave01}
\palavraChaveDois{Palavra-chave02}

% Dados da orientacao
\orientador{<Titulação Acadêmica> Luiz Augusto Laranjeira}
\coorientador{Prof. Dr. Edson Alves Costa Júnior}

% Dados para a ficha catalográfica
\cdu{02:141:005.6}

% Dados da aprovação do trabalho
\dataDaAprovacao{01 de junho de 2013}
\membroConvidadoUm{Titulação e Nome do Professor Convidado 01}
\membroConvidadoDois{Titulação e Nome do Professor Convidado 02}

% Dados pessoais
\autor{Rodrigo Santana Gonçalves e Carlos Alberto Teixeira Junior}
\curso{Engenharia de Software}

% Dados do trabalho
\titulo{Estudo da Segurança Criptográfica De Algoritmos De Curva Elíptica}
\data{2015}
\palavraChaveUm{Palavra-chave01}
\palavraChaveDois{Palavra-chave02}

% Dados da orientacao
\orientador{<Titulação Acadêmica> Luiz Augusto Laranjeira}
\coorientador{Prof. Dr. Edson Alves Costa Júnior}

% Dados para a ficha catalográfica
\cdu{02:141:005.6}

% Dados da aprovação do trabalho
\dataDaAprovacao{01 de junho de 2013}
\membroConvidadoUm{Titulação e Nome do Professor Convidado 01}
\membroConvidadoDois{Titulação e Nome do Professor Convidado 02}

\renewcommand{\theequation}{\arabic{equation}}
\definecolor{blue}{RGB}{41,5,195}
\makeatletter
\hypersetup{
      %pagebackref=true,
    pdftitle={\@title}, 
    pdfauthor={\@author},
      pdfsubject={\imprimirpreambulo},
      pdfcreator={LaTeX with abnTeX2},
    pdfkeywords={abnt}{latex}{abntex}{abntex2}{trabalho acadêmico}, 
    colorlinks=true,          % false: boxed links; true: colored links
      linkcolor=blue,           % color of internal links
      citecolor=blue,           % color of links to bibliography
      filecolor=magenta,          % color of file links
    urlcolor=blue,
    bookmarksdepth=4
}
\makeatother
\setlength{\parindent}{1.3cm}
\setlength{\parskip}{0.2cm}  
\makeindex
\lstset{tabsize=8,numbers=left,literate={:=}{{$\gets$}}1 {<=}{{$\leq$}}1 {>=}{{$\geq$}}1 {<>}{{$\neq$}}1,frame=single,captionpos=b,breaklines=true,mathescape=true,extendedchars=true}

\lstdefinelanguage{pseudo}{
  morekeywords={seja,para,faca,se,retorne,enquanto},
  morekeywords=[2]{inteiro,string,Curva,Ponto},
  sensitive=true,
  morecomment=[l]{\#},
}
\lstset{
 backgroundcolor=\color{white},   % choose the background color; you must add \usepackage{color} or \usepackage{xcolor}
    basicstyle=\footnotesize\ttfamily,
 commentstyle=\color{gray},    % comment style
 numbersep=5pt,                   % how far the line-numbers are from the code
 numberstyle=\tiny\color{black}, % the style that is used for the line-numbers
 rulecolor=\color{Gray},         % if not set, the frame-color may be changed on line-breaks within not-black text (e.g. comments (green here))
 showstringspaces=false,          % underline spaces within strings only
 stringstyle=\color{green},     % string literal style
 tabsize=4,
 numbers=left,
 literate={:=}{{$\gets$}}1 {<=}{{$\leq$}}1 {>=}{{$\geq$}}1 {<>}{{$\neq$}}1,
 frame=single,
 captionpos=t,
 breaklines=true,
 mathescape=true,
 extendedchars=true,
}
%\renewcommand{\thelstlisting}{\arabic{lstlisting}}
% Setup
\lstset{numbers=left,
stepnumber=1,
firstnumber=1,
numberstyle=\tiny,
extendedchars=true,
breaklines=true,
frame=tb,
tabsize=4,
basicstyle=\footnotesize\ttfamily,
stringstyle=\ttfamily,
showstringspaces=false
}

\renewcommand*{\lstlistlistingname}{Lista de Códigos}
\renewcommand{\lstlistingname}{Código}


\begin{document}

\frenchspacing 
\imprimircapa

\textual

\section*{Contextualização}
% Criptografia e suas necessidades

% Curva elíptica e vantagens sobre RSA
A maior parte dos produtos que utilizam a criptografia de chave pública para criptografia e assinaturas digitais utiliza RSA. O tamanho da chave para o uso seguro do RSA tem aumentado nos últimos anos, e isso gerou uma carga de processamento maior sobre as aplicações usando RSA. Esse peso tem ramificações, especialmente para sites de comércio eletrônico, que realizam grandes quantidades de transações seguras. Recentemente, um sistema concorrente começou a desafiar o RSA; criptografia de curva elítica (ECC - Elliptic Curve Cryptography). O ECC já está aparecendo em esforços de padronização, como o IEEE P1363 Standard for Public-Key Cryptography. \cite{Stallings:2011}

O atrativo principal do ECC, em comparação com o RSA, é que ele parece oferecer igual segurança com um tamanho de chave muito menor reduzindo assim o overhead do processamento. Por outro lado, embora a teoria do ECC já exista há algum tempo, só recentemente esses produtos começaram a aparecer, e tem havido interesse criptoanalítico contínuo em encontrar algum ponto fraco. Por conseguinte, o nível de confiança no ECC ainda não é tão alto quanto do RSA. \cite{Stallings:2011}

\section*{Problema de Pesquisa}
% Comparação entre as curvas do NIST e Brainpool

\section*{Justificativa}

\section*{Objetivos}

\begin{enumerate}
	\item \textbf{Objetivo Geral}:

	\item \textbf{Objetivos Específicos}
	\begin{enumerate}
		\item 
	\end{enumerate}
\end{enumerate}

\section*{Metodologia}

\section*{Cronograma}

\postextual

\bibliography{proposta} 

\end{document}
